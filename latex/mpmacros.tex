%
% INDEX MACROS (5th April 1995)
%
% This is rather messy. \index can not be used as part of a command, so
% \mpindex has the same restriction.
%
\newif\ifmpempty
%
% Define \mpTplain as the macro  \mpT
%
\def\mpTplain{\mpT }
%
%
% \mptest
%         Test if #1 is equal to \mpT (the macro)
%         \ifmpempty can be used to check the result .
%
\def\mptest#1{\def\mpTtest{#1}\ifx\mpTtest\mpTplain\mpemptytrue\else%
\mpemptyfalse\fi}
%
%
% INDEX MACROS (release 0.51 onwards)
%
% \mpindexnew has seven arguments. The arguments are parsed to make a correct
% index entry.
%
% Arguments: sortkey, subsortkey, subsubsortkey,
%            visualkey,pagerangestart (, pagenumberfont, pagerangeend )
%
\def\mpindexnew#1#2#3#4#5#6#7{%
\def\myarg{}%
\mptest{#7}\ifmpempty%
  \mptest{#5}\ifmpempty%
    \mptest{#6}\ifmpempty%
      \def\myarg{}%
    \else%
      \def\myarg{|#6}%
    \fi%
  \else%
    \mptest{#6}\ifmpempty%
      \def\myarg{|#5}%
    \else%
      \def\myarg{|#5#6}%
    \fi%
  \fi%
\else%
  \def\myarg{|#7}%
\fi%
\mpmakeentrynew{#1}{#2}{#3}{#4}%
}%
%%%%%%%%%%%%%%%%%%%%%%%%%%%%%%%%%%%%%%%%%%%%%%%%%%%%%%%%
%
%****************************SEE OTHER ENTRY******************************
%
% \mpindexS has five arguments. The last argument should be the entry to
% point to. The first four arguments have the same meaning as in \mpindexnew
%
%  Arguments: sortkey, subsortkey, subsubsortkey,
%             visualkey,see other item.
%
\def\mpindexnewS#1#2#3#4#5{%
\mptest{#5}\ifmpempty%
  \def\myarg{}%
\else%
  \def\myarg{|see{#5}}%
\fi%
\mpmakeentrynew{#1}{#2}{#3}{#4}%
}%
%
% Make an index entry with the four key arguments of \mpindex and \mpindexS
% and add the \myarg value at the end.
%
%%%%%%%%%%%%% version from release 0.51 onwards
\def\mpmakeentrynew#1#2#3#4{%
\mptest{#4}\ifmpempty%
  %no visualcode-----------------------------------------
  \mptest{#2}\ifmpempty%%
    %no sublevel to entry and hence no subsublevel--encap?
    \def\mpentry{#1}%
  \else%
    %sublevel to entry-----------------------------
    \mptest{#3}\ifmpempty%
      %sub but no subsub level to entry--encap?
      \def\mpentry{#1!#2}%
    \else%
      %sub and subsublevel to entry--encap?
      \def\mpentry{#1!#2!#3}%
    \fi%
  \fi%
\else%
  %%%%%%%%%%%%%%%%%%%%%%%%%%%%%%%%%%%%%%%%%%%%%%%%%%%%%%%%%
  %visualcode-----------------------------------------------
  \mptest{#2}\ifmpempty%
    %no sublevel to sortcode and hence no subsublevel--encap?
    \def\mpentry{#1@#4}%
  \else%
    %sublevel to sortcode------------------------------
    \mptest{#3}\ifmpempty%
      %no subsublevel to sortcode--encap?
      \def\mpentry{#1!#2@#4}%
    \else%
      %subsublevel to sortcode---encap?
      \def\mpentry{#1!#2!#3@#4}%
    \fi%
  \fi%
\fi%
\edef\mpfentry{\expandafter\mpstrip\meaning\mpentry\expandafter\mpstrip\meaning\myarg}%
\index{\mpfentry}%
}%
%
%
% INDEX MACROS (release 0.50)
%
% \mpindex has nine arguments. The arguments are parsed to make a correct
% index entry.
%
% This command was written for the first version of the MathSpad manual
% (release 0.50).
%
% The command used from release 0.51 onwards is mpindexnew.
%
% Arguments: sortkey, subsortkey, subsubsortkey,
%            visualkey, visualsubkey, visualsubsubkey,
%            pagerangestart (, pagenumberfont, pagerangeend )
%
\def\mpindex#1#2#3#4#5#6#7#8#9{%
\def\myarg{}%
\mptest{#9}\ifmpempty%
  \mptest{#7}\ifmpempty%
    \mptest{#8}\ifmpempty%
      \def\myarg{}%
    \else%
      \def\myarg{|#8}%
    \fi%
  \else%
    \mptest{#8}\ifmpempty%
      \def\myarg{|#7}%
    \else%
      \def\myarg{|#7#8}%
    \fi%
  \fi%
\else%
  \def\myarg{|#9}%
\fi%
\mpmakeentry{#1}{#2}{#3}{#4}{#5}{#6}%
}
%%%%%%%%%%%%%%%%%%%%%%%%%%%%%%%%%%%%%%%%%%%%%%%%%%%%%%%%
%
%****************************SEE OTHER ENTRY******************************
%
% \mpindexS has seven arguments. The last argument should be the entry to
% point to. The other arguments have the same meaning as in \mpindex
%
%  Arguments: sortkey, subsortkey, subsubsortkey,
%             visualkey, visualsubkey, visualsubsubkey,
%             see other item.
%
\def\mpindexS#1#2#3#4#5#6#7{%
\mptest{#7}\ifmpempty%
  \def\myarg{}%
\else%
  \def\myarg{|see{#7}}%
\fi%
\mpmakeentry{#1}{#2}{#3}{#4}{#5}{#6}%
}
%
% Make an index entry with the six key arguments of \mpindex and \mpindexS
% and add the \myarg valua at the end.
%
\def\mpmakeentry#1#2#3#4#5#6{%
\mptest{#1}\ifmpempty%
  %no sortcode-----------------------------------------
  \mptest{#5}\ifmpempty%
    %no sublevel to entry and hence no subsublevel--encap?
    \def\mpentry{#4}%
  \else%
    %sublevel to entry-----------------------------
    \mptest{#6}\ifmpempty%
      %sub and no subsub level to entry--encap?
      \def\mpentry{#4!#5}%
    \else%
      %sub and subsublevel to entry--encap?
      \def\mpentry{#4!#5!#6}%
    \fi%
  \fi%
\else%
  %%%%%%%%%%%%%%%%%%%%%%%%%%%%%%%%%%%%%%%%%%%%%%%%%%%%%%%%%
  %sortcode-----------------------------------------------
  \mptest{#2}\ifmpempty%
    %no sublevel to sortcode and hence no subsublevel--encap?
    \def\mpentry{#1@#4}%
  \else%
    %sublevel to sortcode------------------------------
    \mptest{#3}\ifmpempty%
      %no subsublevel to sortcode--encap?
      \def\mpentry{#1!#2@#4}%
    \else%
      %subsublevel to sortcode---encap?
      \def\mpentry{#1!#2!#3@#4}%
    \fi%
  \fi%
\fi%
\edef\mpfentry{\expandafter\mpstrip\meaning\mpentry\expandafter\mpstrip\meaning\myarg}%
\index{\mpfentry}%
}
\def\mpstrip#1>{}
%
%
% MISCELLANEOUS MACROS FOR SPEC STENCIL
%
\newcommand{\dstar}{\star}
\newcommand{\bt}{
\begin{tabbing}
\hspace*{\mathindent}\=\kill
}
\def\mpfillbreak{\hfil\break}
\def\mpvskip[#1]{\ignorespaces}
\def\mpvspace[#1]{\vspace{#1}\mpfillbreak\ignorespaces}
\def\mpvs{%
\ifx\mnllet[%
   \ifvmode%
      \let\mnllet=\mpvskip%
   \else%
      \let\mnllet=\mpvspace%
   \fi%
\else%
   \ifvmode%
      \let\mnllet=\empty%
   \else%
      \let\mnllet=\mpfillbreak%
   \fi%
\fi%
\mnllet}
\newcommand{\mnl}{\futurelet\mnllet\mpvs}
\newcommand{\mraisebox}[2]{\mbox{\mathsurround=0pt\raisebox{#1}{#2}}}
\newcommand{\bigsqcap}{\mraisebox{0pt}{\large $\sqcap$}}
\newcommand{\idrelator}{I\!\!I}
\newcommand{\simulates}{\stackrel{>}{\sim}}
\newcommand{\ismodelof}{\stackrel{<}{\sim}}
\newcommand{\et}{\end{tabbing}}
\newcommand{\split}{\mraisebox{.3ex}{{$\scriptscriptstyle \bigtriangleup$}}}
\newcommand{\cosplit}{\mraisebox{.3ex}{{$\scriptscriptstyle \blacktriangle$}}}
\newcommand{\cojunc}{\mraisebox{.3ex}{{$\scriptscriptstyle \blacktriangledown$}}}
\newcommand{\junc}{\mraisebox{.3ex}{{$\scriptscriptstyle \bigtriangledown$}}}
\newcommand{\inl}{{\hookrightarrow}}
\newcommand{\inr}{{\hookleftarrow}}
\newcommand{\lsim}{{\mraisebox{-.25ex}{{\Large ${\sim}$}}}}
\newcommand{\natsim}{\stackrel{.}{\sim}}
\newcommand{\leftrel}{{\mraisebox{.3ex}{{$\scriptscriptstyle <$}}\hspace*{-6.5pt}\sim}}
\newcommand{\rightrel}{{\sim\hspace*{-6.5pt}\mraisebox{.3ex}{{$\scriptscriptstyle >$}}}}
\newcommand{\transform}{\mraisebox{.3ex}{$\scriptscriptstyle <$}\hspace*{-6.5pt}\sim\hspace*{-6.5pt}\mraisebox{.3ex}{$\scriptscriptstyle >$}}
\newcommand{\natrightrel}{\stackrel{.}{\rightrel}}
\newcommand{\natleftarrow}{\stackrel{.}{\leftarrow}}
\newcommand{\natleftright}{\stackrel{.}{\transform}}
\newcommand{\natleftrel}{\stackrel{.}{\leftrel}}
\newcommand{\revbar}{\overline{\rev}}

\newcommand{\revobar}{\revo\!\!\rule{1mm}{0.5mm}\,}
\newcommand{\overbar}{/\!\!\!\rule{1mm}{0.5mm}\,}
\newcommand{\revo}{\backslash}
\newcommand{\scup}{\mraisebox{0.5mm}{$\scriptscriptstyle\cup$}}
\newcommand{\blob}{\mraisebox{0.5mm}{$\scriptstyle\varpi$}}
\newcommand{\eqhat}{\;\:\;\:{\widehat{=}}\;\:\;\:}
\newcommand{\append}{\mathbin{+\!\!\!+}}
\newcommand{\comp}{\mraisebox{.2ex}{$\scriptstyle\circ$}}
\newcommand{\fncomp}{\mraisebox{0.5mm}{$\scriptstyle\bullet$}}
\newcommand{\medfncompose}{\mraisebox{0.5mm}{$\:{\scriptstyle\bullet}\:$}}
\newcommand{\dubarrow}{\mbox{\mathsurround=0pt$\rightarrow\!\!\!\!\!\rightarrow$}}
\newcommand{\dublarrow}{\mbox{\mathsurround=0pt$\leftarrow\!\!\!\!\!\leftarrow$}}
\newcommand{\Real}{{\rm I\!R}}
\newcommand{\Nat}{{\rm I\! N}}
\newcommand{\plus}{+}

\newtheorem{Theorem}[equation]{Theorem}
\newtheorem{Definition}[equation]{Definition}
\newtheorem{Lemma}[equation]{Lemma}
\newtheorem{Axiom}[equation]{Axiom}
\newtheorem{Remark}[equation]{Remark}
\newtheorem{Corollary}[equation]{Corollary}
\newtheorem{Exercise}[equation]{Exercise}
\newtheorem{Example}[equation]{Example}
\newtheorem{defi}[equation]{}

\newcommand{\platbottom}{\bot\hspace*{-4pt}\bot}
\newcommand{\plattop}{\top\hspace*{-4pt}\top}
\newcommand{\ldomsym}{\mraisebox{0.6mm}{$\scriptscriptstyle{<}$}}
\newcommand{\rdomsym}{\mraisebox{0.6mm}{$\scriptscriptstyle{>}$}}
\newcommand{\Ldom}[1]{#1\ldomsym}
\newcommand{\Rdom}[1]{#1\rdomsym}
\newcommand{\leqdomsym}{\mraisebox{0.6mm}{$\scriptscriptstyle{\ll}$}}
\newcommand{\reqdomsym}{\mraisebox{0.6mm}{$\scriptscriptstyle{\gg}$}}
\newcommand{\Leqdom}[1]{#1\leqdomsym}
\newcommand{\Reqdom}[1]{#1\reqdomsym}
\newcommand{\rev}{\scup}

\newcommand{\defeq}{\mraisebox{.4ex}{$\scriptstyle\underline{\bigtriangleup}$}}
\newcommand{\ctilde}{\mraisebox{-.6ex}{${\;\tilde{}\;}$}}
\newcommand{\cross}{\mbox{\mathsurround=0pt\large $\times$}}
\newcommand{\conc}{{+}\!\!\!{+}}
\newcommand{\cbar}{\mbox{--}}
\newcommand{\thinbox}{[\hspace{-1pt}]}
\newcommand{\Bool}{{\rm I\!B}}
\newcommand{\Int}{{\rm Z\hspace{-4pt}Z}}
\newcommand{\1}{1\!\!1}
\newcommand{\paropen}{(\!|}
\newcommand{\parclose}{|\!)}
\newcommand{\dopen}{[\![}
\newcommand{\dclose}{]\!]}
\newcommand{\homl}{\mbox{(\hspace*{-2.2pt}[}}
\newcommand{\homr}{\mbox{]\hspace*{-2.2pt})}}
\newcommand{\lparbrack}{\mathopen{( \mkern -4mu [}}
\newcommand{\rparbrack}{\mathclose{] \mkern -4mu )}}
\newcommand{\sqopen}{|\hspace{-1pt}[}
\newcommand{\sqclose}{]\hspace{-1pt}|}
\newcommand{\lang}{\mraisebox{.5ex}{\scriptsize${<}$}}
\newcommand{\rang}{\mraisebox{.5ex}{\scriptsize${>}$}}
\newcommand{\fst}{\mraisebox{.2ex}{\scriptsize${\ll}$}}
\newcommand{\snd}{\mraisebox{.2ex}{\scriptsize${\gg}$}}

\newlength{\jojo}
\newcommand{\Overlay}[2]{\mathsurround=0pt\settowidth{\jojo}{\mbox{$#1$}}
\makebox[\jojo]{\makebox[0cm]{$#1$}\makebox[0cm]{$#2$}}}
\def\lbrackpar{\mathopen{{{\Overlay{\lfloor}{\lceil}}\mkern-5mu{(}}}}%   % [(
\def\rbrackpar{\mathclose{{{)}\mkern-5mu{\Overlay{\rfloor}{\rceil}}}}}
\newcommand{\Cata}[2]{{\lparbrack {#1} ;~ {#2} \rparbrack}}
\newcommand{\cata}[1]{{\lparbrack {#1} \rparbrack}} 
\newcommand{\Ana}[2]{{\lbrackpar {#1} ;~ {#2} \rbrackpar}} 
\newcommand{\ana}[1]{{\lbrackpar {#1} \rbrackpar}}

%
%  Item separation macros.  (6th December 1995)
%
% The following macros have been written so that users of MathSpad do not 
% themselves have to separate items in a list by ``\\''.  The macro \mpinsertsep
% does it automatically.  Specifically, in a list of items, ``\\\empty'' 
% (``\\'' --the TeX command for a new line-- followed by ``\empty'' --the TeX
% command for an empty string, included in order to avoid problems if the next
% line begins with ``[''-- )
% is inserted *between* consecutive items.  See the
% templates EqnArray, Tabbing, and Array for examples of their use.

\def\mpnewitem{\empty}%
\def\mpinsc{\ifx\mpinspar\mpnewitem\let \mpsep=\\\empty \else \let\mpsep=\empty\fi
 \mpsep}%
\def\mpinsertsep{\futurelet\mpinspar\mpinsc}%

